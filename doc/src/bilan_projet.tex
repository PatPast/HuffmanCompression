\documentclass[a4paper,11pt]{article} 
\usepackage[francais]{babel}
\usepackage[T1]{fontenc} 
\usepackage[utf8]{inputenc} 
\usepackage{graphicx}
\usepackage{color}
\usepackage{hyperref}
 
\usepackage{array}

\usepackage{booktabs}
\usepackage{tabularx}

\title {\textbf {\color {blue} Le Mans Université}\color{black}
\\  Licence Informatique  \textit {3\ieme année}
 \\Module POO en Java
 \\ \textbf {TP2 Arbres -- Bilan de projet}}
\author{\href{mailto: matthieu.boulanger.etu@univ-lemans.fr} {Matthieu \textsc{Boulanger}}\\
      \href{mailto: patrick.pastouret.etu@univ-lemans.fr} {Patrick \textsc{Pastouret}}}
%\\
\date{\today} 

\begin{document}
\maketitle

À la réalisation de ce sujet de travaux pratiques il nous a semblé délicat de choisir la bonne structure de classes pour représenter les AB et ABR. Nous avons décidé assez rapidement de ne faire qu'une classe de n\oe ud qui représente l'arbre tout entier par sa racine, mais percevons rétrospectivement les limites de cette solution. À la lumière des notions introduites dans les cours magistraux plus récents, telles les classes internes, nous envisageons de faire évoluer le modèle initial pour constituer une base plus solide pour la suite du projet.\\
Nous avons hésité entre plusieurs solutions pour gérer de façon générique les arbres de types de valeurs différents avant de nous arrêter sur l'utilisation de l'interface \texttt{Comparable}.

\end{document}
