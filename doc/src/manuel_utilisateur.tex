\documentclass[a4paper,11pt]{article} 
\usepackage[francais]{babel}
\usepackage[T1]{fontenc} 
\usepackage[utf8]{inputenc} 
\usepackage{graphicx}
\usepackage{color}
\usepackage{hyperref}
 
\usepackage{array}

\usepackage{booktabs}
\usepackage{tabularx}

\title {\textbf {\color {blue} Le Mans Université}\color{black}
\\  Licence Informatique  \textit {3\ieme année}
 \\Module POO en Java
 \\ \textbf {TP2 Arbres -- Manuel utilisateur}}
\author{\href{mailto: matthieu.boulanger.etu@univ-lemans.fr} {Matthieu \textsc{Boulanger}}\\
      \href{mailto: patrick.pastouret.etu@univ-lemans.fr} {Patrick \textsc{Pastouret}}}
%\\
\date{\today} 

\begin{document}
\maketitle

\section{Arbres binaires}

Les arbres binaires se créent au moyen de la classe \texttt{BinaryTree}, le constructeur prend comme seul argument la valeur de la racine (de type \texttt{Object}) et les n\oe uds suivants sont ajoutés aléatoirement à l'arbre via la méthode \texttt{addNode(Object valeur)}. \\
Il est possible de les afficher via \texttt{System.out.println} ou toute autre méthode recourant à \texttt{BinaryTree.toString()}.\\
Le test de présence d'une valeur dans l'arbre se fait au moyen de \texttt{exists(Object valeur)}. \\
Le calcul de hauteur se fait via \texttt{getHeight()}.\\
Enfin, une valeur peut être retirée de l'arbre avec \texttt{removeNode(Object valeur)}. La méthode ne supprime que les feuilles. Si la valeur est présente plusieurs fois, toutes les occurrences sont supprimées. Dans l'implémentation courante il n'est pas possible de supprimer la racine même si c'est le dernier n\oe ud restant.

\section{Arbres binaires de recherche}

Les ABR présentent les mêmes méthodes que les AB mais le comportement diffère. Ainsi \texttt{addNode(Object valeur)} insère la valeur selon la structure d'ABR et ignore les doublons. \texttt{exists(Object valeur)} retrouve la valeur par un parcours dichotomique. Enfin, \texttt{removeNode(Object valeur)} permet de supprimer une valeur située sur un n\oe ud non feuille, en conservant la structure d'ABR.

\end{document}
