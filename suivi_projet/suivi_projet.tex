\documentclass[a4paper,11pt]{article} 
\usepackage[francais]{babel}
\usepackage[T1]{fontenc} 
\usepackage[utf8]{inputenc} 
\usepackage{graphicx}
\usepackage{color}
\usepackage{hyperref}
 
\usepackage{array}

\usepackage{booktabs}
\usepackage{tabularx}

\title {\textbf {\color {blue} Le Mans Université}\color{black}
\\  Licence Informatique  \textit {3\ieme année}
 \\Module POO en Java
 \\ \textbf {TP2 Arbres – Suivi de projet}}
\author{\href{mailto: matthieu.boulanger.etu@univ-lemans.fr} {Matthieu \textsc{Boulanger}}\\
      \href{mailto: patrick.pastouret.etu@univ-lemans.fr} {Patrick \textsc{Pastouret}}}
%\\
\date{\today} 

\begin{document}
\maketitle

\section{Objectifs de la séance}
\begin{itemize}
    \item implémenter la classe BinarySearchTree ;
    \item initialiser la mise en forme des livrables.
\end{itemize}

\section{Plan de travail}
\begin{center}
    \begin{tabular}{p{.3\linewidth}ccp{.3\linewidth}}
      \toprule
	Tâche & Responsable & Estimation temps & Commentaire \\
      \midrule
      \bottomrule
   \end{tabular}
\end{center}

\end{document}
1. Objectifs de la Séance
Liste des objectifs à atteindre pour cette séance :
1. Comprendre l’énoncé du TP et analyser les concepts nécessaires.
2. Implémenter la méthode addNode pour ajouter des éléments dans un arbre binaire de
recherche.
3. Tester l’ajout d’éléments et vérifier la structure de l’arbre.
4. Documenter le code avec des commentaires clairs.

2. Plan de Travail
Répartition des tâches entre les membres du binôme :
Tâche
Analyser l’énoncé et
écrire une liste de tâches.
Implémenter la méthode
addNode.
Tester l’ajout d’éléments
dans l’arbre binaire.
Ajouter des commen-
taires au code.
Respons-
able
Étudiant A Estimation
Temps
15 min
Étudiant B 1 heure
Étudiant A 30 min
Étudiant B 15 min
Commentaire
Valider la liste avec l’en-
cadrant.
S’assurer que la méthode
respecte les spécifications.
Utiliser des cas simples
pour commencer.
Respecter les conventions
de documentation.
3. Tâches Accomplies
Tâches réalisées lors de cette séance :
- Méthode addNode implémentée avec succès pour des arbres binaires de recherche.
- Tests réalisés avec des exemples simples.
- Résultat : structure correcte, vérifiée par des impressions console.
- Documentation ajoutée au code (méthodes et classes).
4. Problèmes Rencontrés et Solutions
Problème
Confusion sur la gestion des
doublons lors de l’ajout
dans l’arbre.
Solution Envisagée
Décidé d’ajouter un champ comp-
teur pour chaque nœud.
1
Statut
Résolu.Difficulté à valider l’arbre
après insertion.
Ajouté une méthode pour imprimer
l’arbre en ordre croissant.
Résolu.
5. Objectifs pour la Prochaine Séance
• Implémenter la méthode removeNode pour supprimer des nœuds de l’arbre.
• Mettre en place des tests plus complexes pour vérifier la robustesse de l’arbre.
• Commencer la partie compression de texte avec les arbres.
6. Suivi sur Git
Lien vers le projet GitLab : https://git.univ-lemans.fr/mon-projet
Actions effectuées :
- Création d’un commit pour chaque étape clé :
- “Implémentation de addNode”
- “Ajout de tests unitaires simples pour addNode”
- “Ajout des commentaires au code”
7. Remarques et Questions
• Besoin de clarifications sur la suppression des nœuds complexes (nœuds avec deux
enfants).
• Suggestions pour tester des cas particuliers (exemple : arbres vides, très grands ar-
bres).
2
